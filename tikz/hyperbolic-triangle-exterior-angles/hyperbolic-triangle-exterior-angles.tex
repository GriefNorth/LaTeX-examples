\documentclass[varwidth=true, border=2pt]{standalone}

\usepackage{tkz-euclide}
\usetkzobj{all}
% Thanks to Jan Hlavacek for giving an example
% (http://tex.stackexchange.com/a/16749/5645)

\begin{document}
\begin{tikzpicture}[scale=3]
    \tkzDefPoint(0,0){O}
    \tkzDefPoint(1,0){Z}

    % Define points
    \tkzDefPoint(-0.7,-0.1){A}
    \tkzDefPoint(0.4,-0.3){B}
    \tkzDefPoint(0.1,0.4){C}

    % Draw big circle
    \tkzDrawCircle[fill=white](O,Z)

    % I have no idea what this does,
    % but when I remove it the rest of the image (the big circles)
    % get visible. So it seems to crop the image to the size of the circle
    \tkzClipCircle(O,Z)

    % Draw angles
    \tkzMarkAngles[fill= orange,size=0.3cm,opacity=.3](B,A,C)
    \tkzLabelAngle[pos=0.2](B,A,C){$\alpha$}

    \tkzMarkAngles[fill= orange,size=0.3cm,opacity=.3](C,B,A)
    \tkzLabelAngle[pos=0.2](C,B,A){$\beta$}

    \tkzMarkAngles[fill= orange,size=0.3cm,opacity=.3](A,C,B)
    \tkzLabelAngle[pos=0.2](A,C,B){$\gamma$}


    % "Clip" angles
    \tkzDrawCircle[fill,orthogonal through=A and B,color=white](O,Z)
    \tkzDrawCircle[fill,orthogonal through=A and C,color=white](O,Z)
    \tkzDrawCircle[fill,orthogonal through=B and C,color=white](O,Z)

    % Draw big circle
    \tkzDrawCircle(O,Z)

    % Draw lines
    \tkzDrawCircle[orthogonal through=A and B,color=green](O,Z)
    \tkzDrawCircle[orthogonal through=A and C,color=blue](O,Z)
    \tkzDrawCircle[orthogonal through=B and C,color=red](O,Z)

    % Draw points and label them
    \tkzDrawPoints[color=black,fill=red,size=5](A,B,C)
    \tkzLabelPoints(A,B,C)
\end{tikzpicture}
\end{document}