\documentclass[usepdftitle=false,hyperref={pdfpagelabels=false}]{beamer}

% use KIT-Theme
% see http://sdqweb.ipd.kit.edu/wiki/Dokumentvorlagen
%\usetheme{Frankfurt} % see http://deic.uab.es/~iblanes/beamer_gallery/index_by_theme.html
\InputIfFileExists{templates/beamerthemekit.sty}{\usepackage{templates/beamerthemekit}}{\usetheme{Frankfurt}}
\usefonttheme{professionalfonts}


\usepackage{hyperref}
\usepackage{lmodern}
\usepackage{listings}
\usepackage{wrapfig}        % see http://en.wikibooks.org/wiki/LaTeX/Floats,_Figures_and_Captions

\usepackage[utf8]{inputenc} % this is needed for german umlauts
\usepackage[ngerman]{babel} % this is needed for german umlauts
\usepackage[T1]{fontenc}    % this is needed for correct output of umlauts in pdf

\usepackage{verbatim}
\usepackage{relsize}
\usepackage{subfigure}

% http://en.wikibooks.org/wiki/LaTeX/Algorithms_and_Pseudocode
% http://tex.stackexchange.com/questions/1375/what-is-a-good-package-for-displaying-algorithms
% http://tex.stackexchange.com/questions/26539/beamer-and-pseudocode
% http://www.jkrieger.de/tools/latex/informatik.html
% http://ctan.mackichan.com/macros/latex/contrib/algorithmicx/algorithmicx.pdf
% http://ctan.mackichan.com/macros/latex/contrib/algorithms/algorithms.pdf
% http://www.cs.brown.edu/system/software/latex/doc/algodoc.pdf
% http://www.cs.utexas.edu/~shan/doc/algorithms.pdf
\usepackage{algorithm,algpseudocode}
\usepackage{tikz}
\usetikzlibrary{arrows,shapes,positioning,shadows,calc}
\usepackage{tkz-berge}
\usepackage{xcolor}
\makeatletter

% to change colors
\newcommand{\fillcol}{green!20}
\newcommand{\bordercol}{black}

%%%%%%%%%%%%%%%%%%%%%%%%%%%%%%%%%%%%%%%%%%%%%%%%%%%%%%%%%%%%%%%%%%%%%%
%%%%%%%%%%%%%%%%%%%%%%%%%%%%%%%%%%%%%%%%%%%%%%%%%%%%%%%%%%%%%%%%%%%%%%
% code from Andrew Stacey (with small adjustment to the border color)
% http://tex.stackexchange.com/questions/51582/background-coloring-with-overlay-specification-in-algorithm2e-beamer-package
\newcounter{jumping}
\resetcounteronoverlays{jumping}

\def\jump@setbb#1#2#3{%
  \@ifundefined{jump@#1@maxbb}{%
    \expandafter\gdef\csname jump@#1@maxbb\endcsname{#3}%
  }{%
    \csname jump@#1@maxbb\endcsname
    \pgf@xa=\pgf@x
    \pgf@ya=\pgf@y
    #3
    \pgfmathsetlength\pgf@x{max(\pgf@x,\pgf@xa)}%
    \pgfmathsetlength\pgf@y{max(\pgf@y,\pgf@ya)}%
    \expandafter\xdef\csname jump@#1@maxbb\endcsname{\noexpand\pgfpoint{\the\pgf@x}{\the\pgf@y}}%
  }
  \@ifundefined{jump@#1@minbb}{%
    \expandafter\gdef\csname jump@#1@minbb\endcsname{#2}%
  }{%
    \csname jump@#1@minbb\endcsname
    \pgf@xa=\pgf@x
    \pgf@ya=\pgf@y
    #2
    \pgfmathsetlength\pgf@x{min(\pgf@x,\pgf@xa)}%
    \pgfmathsetlength\pgf@y{min(\pgf@y,\pgf@ya)}%
    \expandafter\xdef\csname jump@#1@minbb\endcsname{\noexpand\pgfpoint{\the\pgf@x}{\the\pgf@y}}%
  }
}

\tikzset{%
  remember picture with id/.style={%
    remember picture,
    overlay,
    draw=\bordercol,
    save picture id=#1,
  },
  save picture id/.code={%
    \edef\pgf@temp{#1}%
    \immediate\write\pgfutil@auxout{%
      \noexpand\savepointas{\pgf@temp}{\pgfpictureid}}%
  },
  if picture id/.code args={#1#2#3}{%
    \@ifundefined{save@pt@#1}{%
      \pgfkeysalso{#3}%
    }{
      \pgfkeysalso{#2}%
    }
  },
  onslide/.code args={<#1>#2}{%
    \only<#1>{\pgfkeysalso{#2}}%
  },
  alt/.code args={<#1>#2#3}{%
    \alt<#1>{\pgfkeysalso{#2}}{\pgfkeysalso{#3}}%
  },
  stop jumping/.style={
    execute at end picture={%
      \stepcounter{jumping}%
      \immediate\write\pgfutil@auxout{%
        \noexpand\jump@setbb{\the\value{jumping}}{\noexpand\pgfpoint{\the\pgf@picminx}{\the\pgf@picminy}}{\noexpand\pgfpoint{\the\pgf@picmaxx}{\the\pgf@picmaxy}}
      },
      \csname jump@\the\value{jumping}@maxbb\endcsname
      \path (\the\pgf@x,\the\pgf@y);
      \csname jump@\the\value{jumping}@minbb\endcsname
      \path (\the\pgf@x,\the\pgf@y);
    },
  }
}


\def\savepointas#1#2{%
  \expandafter\gdef\csname save@pt@#1\endcsname{#2}%
}

\def\tmk@labeldef#1,#2\@nil{%
  \def\tmk@label{#1}%
  \def\tmk@def{#2}%
}

\tikzdeclarecoordinatesystem{pic}{%
  \pgfutil@in@,{#1}%
  \ifpgfutil@in@%
    \tmk@labeldef#1\@nil
  \else
    \tmk@labeldef#1,\pgfpointorigin\@nil
  \fi
  \@ifundefined{save@pt@\tmk@label}{%
    \tikz@scan@one@point\pgfutil@firstofone\tmk@def
  }{%
  \pgfsys@getposition{\csname save@pt@\tmk@label\endcsname}\save@orig@pic%
  \pgfsys@getposition{\pgfpictureid}\save@this@pic%
  \pgf@process{\pgfpointorigin\save@this@pic}%
  \pgf@xa=\pgf@x
  \pgf@ya=\pgf@y
  \pgf@process{\pgfpointorigin\save@orig@pic}%
  \advance\pgf@x by -\pgf@xa
  \advance\pgf@y by -\pgf@ya
  }%
}
\newcommand\tikzmark[2][]{%
\tikz[remember picture with id=#2] #1;}
\makeatother

\resetcounteronoverlays{algocf}

\newcommand<>{\boxto}[1]{%
\only#2{\tikz[remember picture with id=#1]
\draw[line width=1pt,fill=\fillcol,rectangle,rounded corners]
(pic cs:#1) ++(5.2,-.1) rectangle (-0.4,0)
;\tikz\node [anchor=base] (#1){};}% <= insertion to store the anchor to be used as based for the annotation
}
%%%%%%%%%%%%%%%%%%%%%%%%%%%%%%%%%%%%%%%%%%%%%%%%%%%%%%%%%%%%%%%%%%%%%%
%%%%%%%%%%%%%%%%%%%%%%%%%%%%%%%%%%%%%%%%%%%%%%%%%%%%%%%%%%%%%%%%%%%%%%

% Define some styles for graphs
\tikzstyle{vertex}=[circle,fill=black!25,minimum size=20pt,inner sep=0pt]
\tikzstyle{selected vertex} = [vertex, fill=red!24]
\tikzstyle{blue vertex} = [vertex, fill=blue!24]
\tikzstyle{yellow vertex} = [vertex, fill=yellow!24]
\tikzstyle{edge} = [draw,thick,-]
\tikzstyle{weight} = [font=\small]
\tikzstyle{selected edge} = [draw,line width=5pt,-,red!50]
\tikzstyle{ignored edge} = [draw,line width=5pt,-,black!20]

\hypersetup{pdftitle={Graphentheorie II}}
\beamertemplatenavigationsymbolsempty

\newcommand\InsertToC[1][]{
  \begin{frame}{Outline}
    \tableofcontents[subsectionstyle=show/show/show, subsubsectionstyle=show/show/show, #1]
  \end{frame}
}

\begin{document}
\title{Graphentheorie II}
\author{Tobias Sturm, Martin Thoma, Max Wagner, Thomas Krings}
\date{\today}
\subject{Graphentheorie-Referat fur ICPC}

\frame{\titlepage}

\frame{
	\frametitle{Inhaltsverzeichnis}
	\setcounter{tocdepth}{1}
	\tableofcontents
	\setcounter{tocdepth}{2}
}

\AtBeginSection[]{
	\InsertToC[sections={\thesection}]  % shows only subsubsections of one subsection
}

\section{Minimale Spannbäume}

\subsection{Wozu minimale Spannbäume?}
\begin{frame}{Wozu?}{Why?}
	\only<1>{\includegraphics[scale=0.35]{Material/minSpannbaum_1.png}}
	\only<2>{\includegraphics[scale=0.35]{Material/minSpannbaum_2.png}}
	\only<3>{\includegraphics[scale=0.35]{Material/minSpannbaum_3.png}}
	\only<4>{\includegraphics[scale=0.35]{Material/minSpannbaum_4.png}}
	\only<5>{\includegraphics[scale=0.35]{Material/minSpannbaum_5.png}}
\end{frame}

\subsection{Was ist ein minimaler Spannbaum?}
\begin{frame}{Definition}
Minimale Spannbäume sind Teilgraphen, sodass ...
	\begin{itemize}
		\item ... alle Knoten erreichbar sind \pause
		\item ... die Summe der Kantengewichte minimal ist \pause
		\item ... kein Zyklus im Graph enthalten ist ($\Rightarrow$ Baum).
	\end{itemize}
\end{frame}

\begin{frame}{Definition}
	Sei	$G = (V, E) $ mit Kostenfunktion $w: E \rightarrow \mathbb{R}$
	\vspace{10 mm}

	$MST = (V, T)$ ist Spannbaum von G, wenn
	\begin{itemize}
		\item $T \subseteq E$ bzw.
		\item $ \forall u, v \in V: \exists$ Pfad von $u$ nach $v$
		\item $W(T) := \displaystyle\sum\limits_{(u, v) \in T} w(u, v)$ minimal ist.
	\end{itemize}

\end{frame}

\begin{frame}{Eindeutigkeit von Spannbäumen}{Ambiguity of minimal spanning trees}
	Ist dieser Spannbaum eindeutig? \only<2>{Nein}
	\only<1>{\includegraphics[scale=0.35]{Material/minSpannbaum_5.png}}
	\only<2>{\includegraphics[scale=0.35]{Material/minSpannbaum_amb.png}}
\end{frame}

% Author: Kjell Magne Fauske
% Source: http://www.texample.net/tikz/examples/prims-algorithm/
% Declare layers
\pgfdeclarelayer{background}
\pgfsetlayers{background,main}

\subsection{Algorithmus von Prim}
\begin{frame}{Algorithmus von Prim}{Prim's algorithm}
	$S$ ist Menge aller erreichten Knoten, $E$ ist Menge der ausgewählten Kanten.\pause



	Starte bei einem beliebigen Knoten: füge zu $S$ hinzu.
	\begin{enumerate}
		\item wähle Kante am \emph{Rand} von $S$ mit dem geringsten Gewicht und füge zu $E$ hinzu. \pause
		\item füge zugehörigen Knoten zu $S$ hinzu.
		\item Fehlt ein Knoten in $S$ ? goto 1
	\end{enumerate}
\end{frame}

\begin{frame}{Algorithmus von Prim}{Prim's algorithm}
	%% Adjacency matrix of graph
	%% \  a  b  c  d  e  f  g
	%% a  x  7     5
	%% b  7  x  8  9  7
	%% c     8  x     5
	%% d  5  9     x 15  6
	%% e     7  5 15  x  8  9
	%% f           6  8  x 11
	%% g              9  11 x
	\begin{figure}
		\begin{tikzpicture}[scale=1.8, auto,swap]
			% Draw a 7,11 network
			% First we draw the vertices
			\foreach \pos/\name in {{(0,2)/a}, {(2,1)/b}, {(4,1)/c},
				                    {(0,0)/d}, {(3,0)/e}, {(2,-1)/f}, {(4,-1)/g}}
				\node[vertex] (\name) at \pos {$\name$};
			% Connect vertices with edges and draw weights
			\foreach \source/ \dest /\weight in {b/a/7, c/b/8,d/a/5,d/b/9,
				                                 e/b/7, e/c/5,e/d/15,
				                                 f/d/6,f/e/8,
				                                 g/e/9,g/f/11}
				\path[edge] (\source) -- node[weight] {$\weight$} (\dest);
			% Start animating the vertex and edge selection.
			\foreach \vertex / \fr in {d/1,a/2,f/3,b/4,e/5,c/6,g/7}
				\path<\fr-> node[selected vertex] at (\vertex) {$\vertex$};
			% For convenience we use a background layer to highlight edges
			% This way we don't have to worry about the highlighting covering
			% weight labels.
			\begin{pgfonlayer}{background}
				\pause
				\foreach \source / \dest in {d/a,d/f,a/b,b/e,e/c,e/g}
				    \path<+->[selected edge] (\source.center) -- (\dest.center);
				\foreach \source / \dest / \fr in {d/b/4,d/e/5,e/f/5,b/c/6,f/g/7}
				    \path<\fr->[ignored edge] (\source.center) -- (\dest.center);
			\end{pgfonlayer}
		\end{tikzpicture}
	\end{figure}
\end{frame}
%% end of source

\begin{frame}{Algorithmus von Prim}{Prim's algorithm}
%source http://inserv.math.muni.cz/biografie/obrazky/jarnik_vojtech.jpg
%http://www.ams.org/featurecolumn/images/january2006/trees9.jpg
	\textbf{Erfinder}
	\begin{itemize}
		\item 1930: Vojtěch Jarník
		\item 1957: Robert C. Prim
		\item 1959 wiederentdeckt von Edsger Dijkstra
	\end{itemize}

	\begin{figure}
\centering
\mbox{\subfigure{\includegraphics[width=0.6in]{Material/jarnik_vojtech.jpg}}\quad
\subfigure{\includegraphics[width=0.6in]{Material/Prim.jpg} }}
\caption{Jarnik Vojtech und Prim}
\end{figure}
	\textbf{Alternative Bezeichnungen}
	\begin{itemize}
		\item DJP algorithm
		\item Jarník algorithm
		\item Prim–Jarník algorithm
	\end{itemize}

\end{frame}     % Algorithmus von Prim

% Author: Martin Thoma
\subsection{Algorithmus von Kruskal}
\begin{frame}{Algorithmus von Kruskal}{Kruskal's algorithm}
	$E$: Menge der ausgewählten Kanten, $S$: Menge der erreichbaren Knoten.\vspace{10pt}\pause

	So lange, bis alle Knoten erreichbar sind:

	Wähle Kante mit geringstem Gewicht

	Wenn durch ausgewählte Kante ein Knoten erreichbar ist, der davor nicht in $S$ war, füge diese Kante zu $E$ und Knoten zu $E$ hinzu.
\end{frame}


\begin{frame}{Algorithmus von Kruskal}{Kruskal's algorithm}
	\begin{figure}
		\begin{tikzpicture}[scale=1.8, auto,swap]
			% Draw a 7,11 network
			% First we draw the vertices
			\foreach \pos/\name in {{(0,2)/a}, {(2,1)/b}, {(4,1)/c},
				                    {(0,0)/d}, {(3,0)/e}, {(2,-1)/f}, {(4,-1)/g}}
				\node[vertex] (\name) at \pos {$\name$};
			% Connect vertices with edges and draw weights
			\foreach \source/ \dest /\weight in {b/a/7, c/b/8,d/a/5,d/b/9,
				                                 e/b/7, e/c/5,e/d/15,
				                                 f/d/6,f/e/8,
				                                 g/e/9,g/f/11}
				\path[edge] (\source) -- node[weight] {$\weight$} (\dest);
			% Start animating the vertex and edge selection.
			\foreach \vertex / \fr in {d/1,a/1,e/2,c/2,f/3,b/4,g/10}
				\path<\fr-> node[selected vertex] at (\vertex) {$\vertex$};
			% For convenience we use a background layer to highlight edges
			% This way we don't have to worry about the highlighting covering
			% weight labels.
			\begin{pgfonlayer}{background}
				\pause
				\foreach \source / \dest / \fr in {a/d/1,c/e/2,d/f/3,a/b/4,b/e/6,e/g/10}
				    \path<\fr->[selected edge] (\source.center) -- (\dest.center);
				\foreach \source / \dest / \fr in {d/b/5,b/c/7,d/e/8,e/f/9,f/g/11}
				    \path<\fr->[ignored edge] (\source.center) -- (\dest.center);
			\end{pgfonlayer}
		\end{tikzpicture}
	\end{figure}
\end{frame}
%% end of source

\begin{frame}[fragile]
\frametitle{Algorithmus von Kruskal}
\begin{lstlisting}
s is disjunct set of edges
n is number of edges in original graph
while s less than n - 1
e = smallest weight edge not deleted yet
    // edge e = (u, v)
    uset = s.find(u)
    vset = s.find(v)
    if (uset != vset)
        edgesAccepted = edgesAccepted + 1
        s.unionSets(uset, vset)
    end if
end while
\end{lstlisting}
\end{frame}

\begin{frame}{Algorithmus von Kruskal}{Kruskal's algorithm}
	Erfunden von:

	1956: Joseph Kruskal

	\begin{figure}
		\includegraphics[scale=0.6]{Material/kruskal.jpg}
		\caption{Kruskal}
	\end{figure}
\end{frame}
  % Algorithmus von Kruskal % Minimale Spannbäume
\input{SCC}                % Starke zusammenhangskomponenten
\section{Färbung von Graphen}
\subsection{}
\begin{frame}{Färbung von Graphen}{Graph coloring}
	\begin{block}{Problem COLOR}
		Gegeben sei ein Graph $G = (V, E)$ und ein Parameter $K \in \mathbb{N}$.
		Frage: Gibt es eine Knotenfärbung von $G$ mit höchstens $K$ Farben,
		so dass je zwei adjazente Knoten verschiedene Farben besitzen?
	\end{block}
	\begin{itemize}
		\item Ist für 2 Farben entscheidbar (bipartite Graphen)
		\item Für 3 Farben schon $\mathcal{NP}$-vollständig \\
			(Sogar $\mathcal{NP}$-schwer einen 3-färbbaren Graphen mit 4 Farben zu färben)
		\item Für 4 Farben für planare Graphen bewiesenermaßen immer möglich
	\end{itemize}
\end{frame}

\begin{frame}{2-COLOR}{Bipartite Graphen}
	Problem: Gegeben Graph $G=(V, E)$. Ist dieser eine Ja-Instanz von 2-COLOR?

	Lösungsansatz:
	\begin{itemize}
		\item Tiefensuche
		\item Wechsle Farbe nach jedem Knoten
		\item Bei Konflikten breche ab und antworte "Nein"
	\end{itemize}
	Läuft die Tiefensuche ohne abzubrechen durch, ist der Graph bipartit. Aus dem Algorithmus folgt bereits eine gültige Färbung.
\end{frame}

\begin{frame}{3-COLOR}
	Auch hier: Ist Graph $G = (V,E)$ mit 3 Farben färbbar? \\
	Achtung: Problem ist $\mathcal{NP}$-Vollständig.
	\\
	Das heißt es ist kein effizienter Algorithmus bekannt, Laufzeit zur Lösung steigt i.A. exponentiell. \\
	Brute-force für kleine Instanzen des Problems praktikabel. \\
	Für größere Instanzen bietet sich Transformation zu $\mathcal{SAT}$ und Lösung per SAT-Solver an.

\end{frame}
      % Färbung von Graphen
\input{Kreise}             % Euler- und Hamilton-Kreise

\section{Abspann}
\subsection{Abspann}

\begin{frame}{}
Vielen Dank für eure Aufmerksamkeit!
\end{frame}
\begin{frame}{Literatur}
  \begin{itemize}
  \item SCC: Introduction to Algorithms. Second Edition. Thomas H. Cormen. S. 552 - 560.
  \item Minimum Spanning Trees: Introduction to Algorithms. Second Edition. Thomas H. Cormen. S. 561 - 579.
  \item Hamiltonian-cycle problem: Introduction to Algorithms. Second Edition. Thomas H. Cormen. S. 1008 - 1013.
  \item Euler tour: Introduction to Algorithms. Second Edition. Thomas H. Cormen. S. 966.
  \end{itemize}
\end{frame}
          % Quellen und weitere Infos
\begin{frame}{Quellen}
	Quelltext dieser Präsentation auf \href{https://github.com/MartinThoma/ICPC-Referat}{GitHub}

	\textbf{Bilder}
	\begin{itemize}
		\item Jarnik Vojtech:\url{http://inserv.math.muni.cz/biografie/obrazky/jarnik_vojtech.jpg}
		\item Prim: \url{http://www.ams.org/featurecolumn/images/january2006/trees9.jpg}
		\item Kruskal: \url{http://www.cs.umd.edu/~kruskal/kruskal.gif}
		\item Tarjan: \url{http://commons.wikimedia.org/wiki/File:Bob_Tarjan.jpg}
	\end{itemize}

 	\textbf{Tkiz Source}
 	\begin{itemize}
 		\item Prim: \url{http://www.texample.net/tikz/examples/prims-algorithm/}
 	\end{itemize}
\end{frame}


\end{document}
