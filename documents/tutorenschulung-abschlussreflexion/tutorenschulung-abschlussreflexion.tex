\documentclass[a4paper,12pt]{article}
\usepackage{myStyle}

%%%%%%%%%%%%%%%%%%%%%%%%%%%%%%%%%%%%%%%%%%%%%%%%%%%%%%%%%%%%%%%%%%%%%
% Hier eigene Daten einfügen                                        %
%%%%%%%%%%%%%%%%%%%%%%%%%%%%%%%%%%%%%%%%%%%%%%%%%%%%%%%%%%%%%%%%%%%%%
\newcommand{\Jahr}{2012 / 2013}         % Typ: "2011 / 2012" oder "2012"
\newcommand{\Semester}{Wintersemester}  % "Wintersemester" oder "Sommersemester"
\newcommand{\Datum}{\today}             % Wann wurde der Bericht erstellt?
\newcommand{\Ort}{Karlsruhe}
\newcommand{\Studienfach}{Informatik (BA)}
\newcommand{\Institut}{Institut für Theoretische Informatik}
\newcommand{\TutoriumTitle}{Programmieren Tutorium}
\newcommand{\Vorname}{Martin}
\newcommand{\Nachname}{Thoma}
\newcommand{\Matrikelnummer}{1633521}
\newcommand{\Email}{info@martin-thoma.de}
%%%%%%%%%%%%%%%%%%%%%%%%%%%%%%%%%%%%%%%%%%%%%%%%%%%%%%%%%%%%%%%%%%%%%
\hypersetup{
  pdfauthor   = {\Vorname~\Nachname},
  pdfkeywords = {KIT; \Vorname~\Nachname},
  pdftitle    = {Abschlussreflexion von~\Vorname~\Nachname~-~\Semester~\Jahr},
  pdfsubject  = {Tutorenschulung}
}

\pagestyle{fancy}
\fancyhf{}
\renewcommand{\headrulewidth}{0pt}
\renewcommand{\footrulewidth}{0pt}
\fancyfoot[R]{Seite~\thepage~von \pageref{LastPage}}

\definecolor{LightCyan}{rgb}{0.88,1,1}
\definecolor{LightGreen}{rgb}{0.68,1,0.68}
\definecolor{Rosa}{rgb}{1,0.68,0.68}

\pagenumbering{arabic}

\begin{document}
 \pagenumbering{roman}
 % Vorlage aus http://sdqweb.ipd.kit.edu/wiki/Dokumentvorlagen
%% titlepage.tex
%%

% coordinates for the bg shape on the titlepage
\newcommand{\diameter}{20}
\newcommand{\xone}{-15}
\newcommand{\xtwo}{160}
\newcommand{\yone}{15}
\newcommand{\ytwo}{-253}

\begin{titlepage}
% bg shape
\begin{tikzpicture}[overlay]
\draw[color=gray]
 		 (\xone mm, \yone mm)
  -- (\xtwo mm, \yone mm)
 arc (90:0:\diameter pt)
  -- (\xtwo mm + \diameter pt , \ytwo mm)
	-- (\xone mm + \diameter pt , \ytwo mm)
 arc (270:180:\diameter pt)
	-- (\xone mm, \yone mm);
\end{tikzpicture}
	\begin{textblock}{10}[0,0](4,2.5)
		\includegraphics[width=.3\textwidth]{logos/KITLogo_RGB.pdf}
		\hspace{8cm}
	\end{textblock}
	\changefont{phv}{m}{n}	% helvetica
	\vspace*{3.5cm}
	\begin{center}
		\vspace*{3cm}
		\huge{Abschlussreflexion}\\
		\vspace*{1cm}
		\Large{
			\iflanguage{english}{At the Department of Informatics}
					{Tutorenschulung}
		}
	\end{center}
	\vspace*{1cm}
\Large{
\begin{center}
\begin{tabularx}{\textwidth}{@{}llllX}
Name, Vorname    & \Nachname, \Vorname\\
Matrikelnummer   & \Matrikelnummer  \\
Email            & \Email \\
Titel            & \TutoriumTitle\\
Institut         & \Institut\\
Studienfach      & \Studienfach\\
Datum            & \Datum\\
\end{tabularx}
\end{center}
}



\begin{textblock}{10}[0,0](4,16.8)
\tiny{
	\iflanguage{english}
		{KIT -- University of the State of Baden-Wuerttemberg and National Research Center of the Helmholtz Association}
		{KIT -- Universit\"at des Landes Baden-W\"urttemberg und nationales Forschungszentrum in der Helmholtz-Gemeinschaft}
}
\end{textblock}

\begin{textblock}{10}[0,0](14,16.75)
\large{
	\textbf{www.kit.edu}
}
\end{textblock}

\end{titlepage}

 \setcounter{page}{2}
 \pagenumbering{arabic}

% \title{Semesterbericht über das \Semester \Jahr}
% \author{\Vorname \Nachname}
% \date{\Datum}

\section*{Meine Lehre und ihr Umfeld}
Das Programmieren-Tutorium habe ich jeden Montag von 15:45 bis 17:15
Uhr gehalten. Insgesamt habe ich so an 14 Tagen meinen 19 Studenten
Grundlagen in Java beigebracht. Allerdings waren von den 19 Studenten
nur etwa sieben regelmäßig anwesend.

Das Ziel meines Tutoriums war es, alle Studenten in die Lage zu
versetzen, einfache Programme schreiben zu können. Außerdem sollten
sie wissen, wie sie sich neues Wissen im Bereich Programmieren
aneignen. Dadurch sollten sie in der Lage sein, die Abschlussaufgaben
mit einer guten Note zu bestehen.\\
Meine Studenten haben vermutlich genau das von mir erwartet. Außerdem
hatte ich den Eindruck, dass sie von mir erwarteten, dass ich den
Vorlesungsstoff wiederhole. Da von den anwesenden Studierenden leider
sehr viele keinen Computer im Tutorium hatte, habe ich mich dieser
Erwartungshaltung gebeugt.

Wenn ich Fragen zur Lehre hatte, habe ich mit anderen Tutoren, dem
Übungsleiter und meinen Studenten geredet. Bei inhaltlichen Fragen
habe ich auf \href{http://stackoverflow.com/}{stackoverflow.com}
zurückgegriffen.
Die Betreuung des Übungsleiters war gut. Er war immer gut per Email
zu erreichen und ist sehr zügig auf Fragen eingegangen.

Frontalunterricht - das Format meiner Wahl für mein Tutorium - schien
mir aus einigen Gründen passend: Zum einen waren meine Studenten
nicht besonders motiviert, an dem Tutorium teilzunehmen. Viele sind
bereits müde in das Tutorium gekommen und ich sehe es nicht als meine
Aufgabe an, die Studenten zu bespaßen oder zur Teilnahme zu
motivieren. Zum anderen hatten die meisten keinen Computer, was
anspruchsvollere Praxisaufgaben unpraktikabel macht. Ein weiterer
Grund für Frontalunterricht war die Erwartungshaltung der Studenten,
dass ich den Stoff wiederhole.

Neuen Programmieren-Tutoren kann ich keine überaschenden Tipps geben:
Schaut euch die Folien von anderen Tutoren an (z.B. meine unter
\href{martin-thoma.com/programmieren-tutorium}{martin-thoma.com}),
erledigt eure Korrekturen direkt nach der Abgabe, bereitet euch gründlich
auf das Tutorium vor.

\section*{Ich als Lehrperson und mein didaktisches Handeln}
Zu beginn meines Tutoriums habe ich mich als Lehrer gesehen, der
den Studenten helfen soll, Lücken in bereits vorhandenem wissen zu
schließen. Ich habe jedoch relativ schnell das Gefühl gehabt, dass
ich manchen Personen nur beim Einschlafen helfen soll und andere den
Stoff zum ersten mal hören.

Um das Tutorium interessant zu halten, habe ich häufig darauf
hingewiesen, was in der Praxis bzw. in der Wirtschaft tatsächlich
verwendet wird und was man nur für die Abschlussaufgaben braucht.
Da ich schnell mitbekommen habe, dass wohl mehr Studenten meine
Folien anschauen als tatsächlich in meinem Tutorium anwesend sind,
habe ich diese sehr ausführlich formuliert. Somit sollte es den
Studenten allein mit den Folien gut möglich sein, den Stoff zu lernen.
Auch habe ich am Ende häufiger auf kleine Programmierprojekte, wie
z.B. das Programmieren der Spiele Snake, Tetris oder Sokuban hingewiesen
und ausführliche Tutorials in Form von Links bereitgestellt.
Nach dem sechstem Tutorium habe ich eine anonyme Feedback-Runde gemacht.
Dabei haben mir meine Studenten bestätigt, dass sie den Aufbau meines
Tutoriums für gut strukturiert halten, meine Folien äußerst hilfreich
sind und ich sowohl von der Lautstärke, als auch in bezug auf die
Geschwindigkeit gut rede. Außerdem haben sie sich eher mehr
Frontalunterricht und weniger Praxisaufgaben gewünscht. Die
Java-Quizze - kurze Code-Fragmente, die mindestens eine problematische
Stelle haben, die die Studenten finden sollen - sind äußerst positiv
angekommen.

Die größte fachliche Herausforderung waren unbounded Wildcards im
Themenkomplex "`Generics"'. Da dies allerdings nicht Prüfungsrelevant
für die Studenten ist, gab das keine Probleme. Ich habe den
Studenten gesagt, dass ich mir nicht sicher bin, ob ich das Verhalten
dieser in konkreten Detailfragen richtig beantworten kann. Für den
Fall das jemand daran interessiert sein sollte, habe ich die
relevanten Stellen in der Spezifikation bereitgestellt, Links auf
Tutorials einfließen lassen und auf die Experten bei StackOverflow.com
hingewiesen.

\section*{Persönlicher Qualifizierungsprozess}
Aus dem Tutorenprogramm habe ich mitgenommen, dass ich die Arme nicht
so häufig verschränken sollte.

Abgesehen von der Fachlichen kompetenz, die man immer ausbauen kann,
sehen weder meine Studenten noch ich Verbesserungsmöglichkeiten.

\section*{Lehrhospitation}
Bei der kollegialen Lehrhospitation, die ich mit Nilan Marktanner
durchgeführt habe, ist mir aufgefallen, dass er deutlich mehr
Praxisaufgaben macht, dafür aber weniger Frontalunterricht durchführt.
Mir persönlich hätte es auch besser gefallen, mit meinen Studenten
mehr Praxisaufgaben zu machen. Da die Studenten aber weder dazu
bereit waren, noch - aufgrund der fehlenden Computer - dazu in der
Lage waren, ist das bei mir schlicht nicht möglich gewesen.

Es hat mich außerdem sehr gefreut zu erfahren, dass den Studenten
anderer Tutorien auch meine Folien bzw. meine Hilfsmaterialien
bekannt sind und die Folieninhalte weiterverwendet werden.

\clearpage
\section*{Ablaufplanung für eine Einzelveranstaltung}
\textbf{Datum und Thema}: 14. Januar 2013, Sortieren, equals,
hashCode(), abstrakte und finale Klassen\\
\textbf{Lernziele}: Die Studenten sollten nach dem Tutorium \dots
\begin{itemize}
    \item in der Lage sein Liste und Arrays in Java zu sortieren,
    \item die Bedeutung der Methode hashCode() kennen,
    \item wissen, wann man Interfaces, abstrakte Klassen und finale
          Klassen verwendet.
\end{itemize}

\begin{tabularx}{\textwidth}{@{}llllX}
Dauer   & Inhalte        & Methoden            & Medien  & Sinn \& Zweck\\
\hline
8 min   & Java-Quiz      & \parbox[t]{2cm}{Blitzlicht\\Umfrage} & Folien  & Studenten treffen ein, können sich auf das Tutorium einstellen, bekommen eine Einleitung ins Thema und werden direkt eingebunden\\
3 min   & Quiz-Auflösung & Frontal  & Folien    & Studenten sehen das Problem im Quiz\\
2 min   & Altes Übungblatt & Frontal& Folien    & Erklärung zu einer Frage aus dem letzten Tutorium\\
5 min   & Nachtrag zu equals() &Frontal & Folien& instanceOf vs. getClass()\\
25 min  & Sortieren      & Frontal  & \parbox[t]{2cm}{Folien\\Tafel}    & Studenten sollen sortieren können\\
30 min  & hashCode()     & Frontal  & \parbox[t]{2cm}{Folien\\Tafel}    & Studenten sollen die Aufgabe auf dem nächsten ÜB verstehen\\
 2 min  & Interface      & Frontal  & Folien    & Wiederholung\\
 7 min  & abstrac class  & Frontal  & Folien    & Ergänzung zur Vorlesung\\
 3 min  & final class    & Frontal  & Folien    & Ergänzung zur Vorlesung\\
 5 min  & Ende           & Einzelgespräch  & -  & Studenten können Fragen stellen\\
\end{tabularx}

\textbf{Kurzreflexion}:

Die Umsetzung dieser Planung hat gut funktioniert. Im nächsten
Übungsblatt haben alle Studenten korrekt sortiert und die meisten
eine sinnvolle Hash-Funktion erstellt. Da ich bei der Erstellung der
Folien mir natürlich vorher überlege, was ich den Studenten beibringen
will, haben die Lernziele geholfen. Allerdings ist mir auch
nicht klar, wie man guten Frontalunterricht ohne Lernziele überhaupt
machen sollte.
Ich habe bei dem Tutorium nicht so konkret die Zeiten aufgeschrieben.
Ich habe es im Gefühl, wie lange die Studenten für den Stoff brauchen
und das so gelegt, wie es mir als passend erschien. Das hat - wie immer -
wunderbar geklappt. Da ich für die Vorbereitung der Folien jeweils
über 9 Studen investiert habe, ist es vielleicht nicht ganz so erstaunlich,
dass ich keinen konkreten Zeitplan brauche, um die benötigte Zeit
abschätzen zu können.

Es gab im Ablauf keine unvorhergesehenen Situationen.

\noindent \Ort, den \Datum\\
\\
\includegraphics[height=10mm]{max-mustermann.pdf}\\
\Vorname~\Nachname
\end{document}
