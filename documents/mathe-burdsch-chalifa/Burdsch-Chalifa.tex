\documentclass[a4paper,9pt]{scrartcl}
\usepackage[ngerman]{babel}
\usepackage[utf8]{inputenc}
\usepackage{amssymb,amsmath}
\usepackage{geometry}
\usepackage{graphicx}
\usepackage{hyperref}
\usepackage{xcolor}
\definecolor{sky}{HTML}{AAEEEE}
\definecolor{lgreen}{HTML}{77DD11}
\usepackage{tikz}
\usetikzlibrary{calc, decorations.pathmorphing, decorations.text}

\geometry{a4paper,left=18mm,right=18mm, top=1cm, bottom=2cm}

\setcounter{secnumdepth}{2}
\setcounter{tocdepth}{2}

\shorthandon{"}
\hypersetup{
    pdftitle={Sichtweite des Burdsch Chalifa},
    pdfsubject={Aufgabe},
    pdfauthor={Martin Thoma},
    pdfkeywords={Aufgabe, Mathematik, Geometrie, Rechenweg, Lösung}}
\shorthandoff{"}

\begin{document}
 \title{Sichtweite des Burdsch Chalifa}
 \author{Martin Thoma}

 \setcounter{section}{1}
 \section*{Aufgabenstellung}
    Der Burdsch Chalifa war 2010 das höchste Gebäude der Erde. Bis zur Spitze
    sind es 830 m.\\
    \\
    Angenommen, die Erde wäre eine perfekte Kugel mit einem Radius von 6370 km
    und die Sicht wäre nicht durch Nebel, Wolken oder sonstige Hindernisse
    eingeschränkt. Aus welcher Entfernung, die man über die Erde direkt zum
    Burdsch Chalifa zurücklegt, könnte man den Burdsch Chalifa maximal sehen?

    \subsection{Situationsskizze}


\begin{tikzpicture}[scale=1.2]
    \draw[fill=sky] (0, 0) circle (1.5cm);
    \draw[fill=brown] (0, 0) circle (1cm);

    %tangente
    \begin{scope}[shift={(-1.35cm,0)}, rotate=48]
        \draw (0,0) -- node {} (2.5cm, 0);
    \end{scope}

    \draw (0,0) -- node {} (0, 1.5cm);
    \draw (0,0) -- node[anchor=east] {$r$} (138:1cm);

    % angle
    \draw[fill=gray!30] (0,0) -- (138:0.75cm) arc (138:90:0.75cm);
    \draw (110:0.5cm) node {$\varphi$};

    % winkelbogen
    \draw[lime] (138:1cm) arc (138:90:1cm);
    \node[lime] at (-0.2,0.85) {$x$} ;

    % height
    \draw[blue] (0,1cm) -- node[anchor=west] {$h$} (0, 1.5cm);

    \node at (0,-0.2) {Erde} ;

\end{tikzpicture}


    Gesucht ist die Länge des neongrün hervorgehobenen Kreisbogens x.

    \subsection{Rechenweg}
    \begin{align}
        x &= \text{Umfang} \cdot \frac{\phi}{360^\circ} \\
          &= 2 \cdot r \cdot \pi \cdot \frac{cos^{-1}(\frac{r}{r+h})}{360^\circ} \\
          &= 2 \cdot 6370 \text{km} \cdot \pi \cdot \frac{cos^{-1}(\frac{6370}{6370,83})}{360^\circ} \\
          &= 102,8 \text{km}
    \end{align}

    \subsection{Antwort}
        Bei optimalen, also unrealistischen, Bedingungen könnte man die Spitze
        des Burdsch Chalifa noch in 102,8 km entfernung sehen. Dies entspricht
        übrigens auch dem Punkt auf der Erdoberfläche, der vom Burdsch Chalifa
        am weitesten entfernt und zu sehen ist.\\
        Auch wenn nur die Luftlinie gemessen wird, sind es 102,8 km, da der
        Erdradius bedeutend größer als der Burdsch Chalifa ist.\\
        \\
        Laut Bildzeitung kann man die Spitze des Burdsch Chalifa noch in 95 km
        sehen\footnote{http://www.bild.de/lifestyle/bams/burj-chalifa/burj-chalifa-bei-dieser-story-wurde-uns-schwindelig-828-meter-11056462.bild.html vom 10. Juni 2010. Abgerufen am 28. Mai 2011.}

 \section{Erweiterung der Aufgabenstellung}
    Das Dorf Mileiha liegt direkt östlich vom Burdsch Chalifa
    (25$^\circ$ 11' 50'' N, 55$^\circ$ 16' 27'' O). \\
    Wie weit östlich darf das Dorf maximal liegen, damit man die Spitze des
    Burdsch Chalifa bei optimalen Bedingungen noch sehen kann?\\
    Hinweis: Es gelten noch immer die gleichen Voraussetzungen wie im ersten
    Teil der Aufgabe.

    \subsection{Situationsskizze}
\begin{tikzpicture}[scale=3.5,dot/.style={circle,fill=black,minimum size=4pt,inner sep=0pt,
            outer sep=-1pt}]
    \draw[fill=sky!20] (180:1.2cm) arc (180:0:1.2cm);
    \draw[fill=brown!20] (180:1cm) arc (180:0:1cm);
    \draw (-1.2cm,0) -- node {} (1.2cm, 0);
    \draw (0,0) -- node {} (0, 1cm);
    \node at (0,1.1cm) {N} ;

    \coordinate (DEnd) at (0.867cm,0.5cm) ;
    \draw[dashed] (-0.86cm,0.5cm) -- node[above] {} (DEnd);

    \node[above] at (-0.4cm, 0.45cm) {25. Breitengrad};
    \node[above, text width=1cm] at (0.15cm, 0.5cm) {\scriptsize Burdsch\\Chalifa};
    \node[above, text width=1cm] at (0.7cm, 0.5cm) {\scriptsize Mileiha};

    \path [decorate,
            decoration={
                text along path,
                text={6370 km},
                text align={align=center}
            }
          ] (0,0) -- (DEnd);
    \draw (0,0) -- node {} (DEnd);
    \draw[arrows={stealth-stealth}] (-1cm,-0.05cm) -- node[below] {6370 km} (0,-0.05cm);

    \node[dot, lgreen] at (0.2cm,0.5cm) (S) {};
    \node[dot, lgreen] at (0.5cm,0.5cm) (E) {};



    \draw[lgreen, thick] (S) -- node {} (E);

    \begin{scope}[shift={(DEnd)}]
        \draw[fill=gray!20] (0,0) -- (180:0.3cm)  arc (180:210:0.3cm);
        \draw (195:0.2cm) node {$\scriptstyle 25^\circ$};
    \end{scope}



\end{tikzpicture}

        Gesucht ist die grün eingezeichnete Kurve, die sich über die
        Erdoberfläche krümmt. Ihre Länge sei x.\\
        Um diese zu berechnen, müssen wir wissen welchen Radius die Kreisfläche
        hat, die entsteht, wenn man die Erde am 25. Breitengrad schneidet. Der
        Radius dieser Kreisfläche sei $r_{25}$.
    \subsection{Berechnung}
        \begin{align}
            \text{Breitengrad} &= 25 + \frac{11}{60} + \frac{50}{60 \cdot 60} \\
            \text{Breitengrad} &= \frac{9071}{360} \approx 25,1972 \\
            cos(\frac{9071}{360}) &= \frac{r_{25,1972}}{6370\text{km}} \\
            r_{25,1972} &= cos(\frac{9071}{360}) \cdot 6370\text{km} \\
            r_{25,1972} &\approx 5764\text{km}
        \end{align}
        Der soeben errechnete Radius kann einfach in die im ersten Abschnitt
        erarbeitete Formel eingesetzt werden:
    \begin{align}
        x &= 2 \cdot r \cdot \pi \cdot \frac{cos^{-1}(\frac{r}{r+h})}{360^\circ} \\
          &= 2 \cdot 5764 \text{km} \cdot \pi \cdot \frac{cos^{-1}(\frac{5764}{5764,83})}{360^\circ} \\
          &\approx 97,8 \text{km}
    \end{align}
        Nun sollte man noch berücksichtigen, dass die Beobachter wohl nicht auf
        der Erde kriechen, sondern ihre Augen in einer Höhe von ca. 1,6m sind:
    \begin{align}
        x &= 2 \cdot 5764 \text{km} \cdot \frac{\pi}{360^\circ} \cdot ( cos^{-1}(\frac{5764}{5764,83}) + cos^{-1}(\frac{5764}{5764,0016}) \\
          &\approx 102 \text{km}
    \end{align}

    \subsection{Antwort}
        Der am weitesten entfernte Punkt, der direkt östlich vom Burdsch Chalifa
        steht und von dem aus die Spitze des Burdsch Chalifa unter optimalen
        Bedinungen noch erkannt werden kann, liegt ca. 102 km entfernt. \\
        \\
        Anmerkung: Mileiha liegt ca. 60 km vom Burdsch Chalifa entfernt. Er
        müsste also von Mileiha zu sehen sein.
\end{document}
